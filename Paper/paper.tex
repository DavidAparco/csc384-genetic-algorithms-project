\documentclass[letterpaper]{article}
\usepackage{aaai}
\usepackage{times}
\usepackage{helvet}
\usepackage{courier}
\usepackage{url}
\usepackage[table]{xcolor}
\usepackage{graphicx}
% %%%%%%%%%%%%%%%%%%%%%%%%%%%%%%%%%%%%%%%%%%%%%%%%%%%%%%
% PDFMARK for TeX and GhostScript
% Uncomment and complete the following for metadata if
% your paper is typeset using TeX and GhostScript (e.g
% if you use .ps or .eps files in your paper):
% \special{! /pdfmark where
% {pop} {userdict /pdfmark /cleartomark load put} ifelse
% [ /Author (John Doe, Jane Doe)
% /Title (Paper Title)
% /Keywords (AAAI, artificial intelligence)
% /DOCINFO pdfmark}
% %%%%%%%%%%%%%%%%%%%%%%%%%%%%%%%%%%%%%%%%%%%%%%%%%%%%%%
% PDFINFO for PDFTeX
% Uncomment and complete the following for metadata if
% your paper is typeset using PDFTeX
\pdfinfo{
  /Title (Solving an Exam Scheduling Problem Using a Genetic Algorithm)
  /Subject (Input The Proceedings Title Here)
  /Author (Dave, Kordalewski;
           Caigu, Liu;
           Kevin, Salvesen;)
} 
% %%%%%%%%%%%%%%%%%%%%%%%%%%%%%%%%%%%%%%%%%%%%%%%%%%%%%%
% Uncomment only if you need to use section numbers
% and change the 0 to a 1 or 2
% \setcounter{secnumdepth}{0}
% %%%%%%%%%%%%%%%%%%%%%%%%%%%%%%%%%%%%%%%%%%%%%%%%%%%%%%

\title{Solving an Exam Scheduling Problem Using a Genetic Algorithm}
\author{Dave Kordalewski \\ davekordalewski@gmail.com
   \And Caigu Liu \\liucaigu@gmail.com
   \And Kevin Salvesen \\ kevin.salvesen@gmail.com}
   
\nocopyright

\begin{document}
\maketitle

\begin{abstract}
  We examine the exam scheduling problem with soft constraints for problems of varying
  sizes using a genetic algorithm, random search, hillclimbing, and a variant of simulated
  annealing. This is done in the context of maximizing the fitness of a schedule, where the
  fitness is relatively computationally expensive to compute for any schedule. The genetic
  algorithm is discussed in detail. We consider selecting parameters for the GA that
  maximize the quality of the schedule found after a constant number of evaluations. The
  GA, with a naive crossover operator, performs much better than the random search, but
  simulated annealing is far superior in this search space.
\end{abstract}

\section{The Problem}
  The problem we investigate is a sort of scheduling problem. 
  We are attempting to find the most satisfactory choice of when
  and where to hold exams, given a background of student course loads. 

  An instance of this scheduling problem consists of a number of days
  on which exams can be scheduled ($d$), the number of time slots in which
  an exam can be scheduled on any day ($t$), a set of rooms ($R$), a set of
  courses ($C$), and a set of students ($S$), each of whom ($s$) has a
  particular course load, some subset of the courses ($L_s$).

  A schedule, then, is a mapping from courses to rooms at times, 
  which we can express like this:
  
  \[ C \rightarrow \left\{(r,\ a,\ b)\ |\ r\in R,\ a\in \{1..d\},\ b\in \{1..t\}\right\} \]
  
  The total number of different possible schedules in any problem instance is
  $|C|^{|R|dt}$ which is typically far too large for brute force search.
  For example, one instance that we examine (scheduling the fall semester
  exams at the University of Toronto in 2009) involves 603 courses, 7 days,
  8 times, 43 rooms, and 21945 students. This instance admits approximately 
  $10^{6695}$ different possible schedules.

  Typically, with relatively loose constraints on the number of rooms and 
  particular schedules of students, it is not difficult to find some consistent
  schedule; that is, one where no student is asked to write 2 exams
  simultaneously and no room has 2 exams occur in it simultaneously.

  Rather than worry about hard constraints, we prefer a framework of soft 
  constraints, where we try to find a schedule that makes the students and 
  invigilators happiest overall, allowing the possibility that some student 
  or room is left with an impossible exam schedule, which can, in the real 
  world, be dealt with on an individual basis.
  
  The timetable that a student $s \in S$ has under some particular schedule 
  $K$ may be represented in this way:
  
  \[ TT_s(K) = \left\{(a, b)\ |\ \exists r \in R, \exists c \in L_s, K(c)=(r, a, b) \right\} \]
  
  which may, in general, be a multiset.
  
  Similarly, the timetable for a room $r \in R$ is
  
  \[ TT_r(K)=\left\{(a, b)\ |\ \exists c\in C,\ K(c)=(r, a, b)\right\} \]
  
  We define two quality functions, mapping schedules to real numbers in [0,1],
  one for students and one for rooms. These are meant to capture how much they
  "like" the schedule under consideration. (For instance, a student will rate
  poorly any schedule where she has 2 consecutive exams, and very poorly any 
  schedule which expects her to take two exams at the same time.)
  
  \[ Q_s(K)=q_{student}\left(TT_s(K)\right):\textit{ Schedules}\rightarrow [0,\ 1] \]
  \[ Q_r(K)=q_{room}\left(TT_r(K)\right):\textit{ Schedules}\rightarrow [0,\ 1] \]
  
  We will consider later how to define these functions in a reasonable way.

  The quality of a schedule, then, is given by
  
  \[ Q(K)= \frac{\left(w_s\left(\sum_{s \in S}{\frac{Q_s(K)}{|S|}}\right) +
     w_r\left(\sum_{r\in R}{\frac{Q_r(K)}{|R|}} \right)\right)}{(w_s+w_r)} \]
  
  where $w_s$ and $w_r$ are the weightings we can use to indicate that we 
  care somewhat more about students preferences than rooms, giving a quality
  (or fitness, in the terminology of genetic algorithms) in the range [0,1] for
  any schedule.
  
  Evaluating this function Q can be computationally expensive when $|S|$ and $|R|$
  are large. Our work examines how to find a schedule with relatively high fitness
  considering that we will want to do this with as few evaluations of Q as possible.

  We examine, first and foremost, a genetic algorithm approach to solving this problem,
  but also touch on a few other methods.

\section{Approach and Implementation}
  \subsection{Differences From Natural Exam Scheduling Problems}
    The problem we have constructed here is a compromise between the most
    mathematically pure problem with the same sort of properties (which might ignore the
    rooms and related constraints completely) and the problem that must be solved in the
    real world when scheduling thousands of students, which has some important differences
    from this problem.
    
  \subsection{Tools used}
    The program is implemented in Java, to generate possible schedules and to calculate
    their fitness. A Mathematica program is also written to plot graphs, given the data
    generated by the Java program. Additionally, we use Google code 
    subversion\footnote{\url{http://code.google.com/p/csc384-genetic-algorithms-project/}}
    and Google doc to share the work.
    
  \subsection{Fitness Function}
    In order to assign each schedule a fitness, we need to define the functions 
    $q_{student}$ and $q_{room}$. They need to capture the idea of determining how
    much each student or room likes their given timetable. We do this by calculating 
    the penalty incurred by a room, which is a measure of it having properties that 
    are unpleasant to the student or room, and then using the formula
    
    \[ q(TT)=\frac{1}{1+penalty_{TT}} \]
    
    This function will be in the desired range or [0,1].
    
    For students, we accrue a penalty of 50 for every time that is repeated in a 
    timetable, 5 when there are exams in 2 consecutive times, 3 when 2 exams happen 
    on the same day, and 0.5 when the student has exams on consecutive days. This 
    accords reasonably well with real students' hopes for their exam timetables.
    
    Rooms have the same penalty of 50 when it is used by two exams at the same time, 
    5 if it hosts exams in consecutive time slots (reasoning that the invigilators 
    need time to let the students out and in and prepare for the next exam) and a 
    penalty of 0.5 when a room is empty for an entire day between uses (reasoning 
    that the room requires some amount of modifications to be used for exams, and 
    it is difficult to use it for other purposes if it will still be used for exams 
    soon.)
    
    The details of the function we used are essentially arbitrary.
    
  \subsection{Genetic Algorithm Search}
    Although there are many readily available open source genetic algorithm software
    packages available (see JGAP, Jenes) we 
    decided to write a simple, yet general, genetic algorithm package, in Java, that
    can interface with data types designed by others, so long as they implement a few
    necessary methods. The package we wrote is more than adequate for investigating
    the scheduling problem under consideration here.
    
    \subsubsection{General Description}
      The genetic algorithm maintains a population of schedules and constructs subsequent
      generations by adding new random schedules, making copies of high quality schedules 
      in the present generation, applies the mutation operator to schedules, or uses the
      crossover operator on 2 schedules. When performing mutation or crossover, the 
      schedules are selected with a probability proportional to their fitness, allowing 
      the schedules with high quality to be selected more frequently and make more 
      "offspring". 
      
      The free parameters in our GA are the size of the population, number of
      generations to simulate, and proportions of new generations to be made by the copy, 
      random, mutation, and crossover methods.
      
    \subsubsection{Representation}
      We use the naive representation of schedules. A schedule contains lists of size $|C|$
      containing the room in which each course's exam is to occur and the timing
      (corresponding to a pair or integers representing the day and time) of each course's
      exam.
      
    \subsubsection{Mutation Operator}
      The mutation operator takes a probability and a schedule. We loop over the courses,
      and, with the given probability, each is or is not set to a random room at a random time.
      
    \subsubsection{Crossover Operator}
      Given two schedules, the crossover operator loops over the courses, and chooses at
      random which schedule to take the room and timing from for that course. This results 
      in an interleaving of the original courses' data.
      
      There is, unfortunately, no good reason to expect that the result of applying 
      crossover to two relatively high quality schedules will produce a high quality 
      schedule, since the quality of schedules can be very subtle, as they are a more 
      global property of the schedule than this process captures. The crossover of two 
      schedules where no student or room has a timing conflict is unlikely to result in 
      a schedule with much more than average quality. There is no obvious reason to think 
      that this sort of procedure will produce schedules with better than random fitness. 
      Luckily, the schedules produced in practice are better than random.
      
      If a different representation were chosen, such that the mutated schedules were
      different from, but likely to have quality similar to their parents, the GA would 
      likely be able to perform much better than it does, but we could not discover such a
      representation and are satisfied with this method for present purposes.
      
  \subsection{Other Search Algorithms}
    \subsubsection{Random Search}
      The random search method simply generates random schedules, evaluates their fitness,
      and remembers the one that it has seen that has highest fitness. This method is meant
      as a baseline against which to compare the other methods. If a search method doesn't
      perform significantly better than random search with the same number of fitness
      evaluations, it is useless.
      
    \subsubsection{Hillclimbing Search}
      The hillclimbing search that we have implemented examines every schedule in the
      neighbourhood of a given starting schedule, and then continues from the one with
      highest fitness, terminating when it has performed as many evaluations as it is allowed
      to, or it has found a schedule whose neighbour are all equal or inferior to it.
      
      The neighbouring schedules are those where only one course meets in either a different
      room, on a different day, or at a different time. This leads to a total of
      $ |C|(d+t+|R|-3) $ schedules in the neighbourhood of any given schedule.
      
      It can be seen that this method is completely deterministic given a starting schedule,
      and will always return a local maximum

    \subsubsection{Mutate Search}
      Mutate Search starts by generating a random schedule. It makes an alternative schedule
      (by using the same mutate operator made for the genetic algorithm) with 1\% of the
      data randomized and chooses which of the schedules is better. This operation is
      repeated until it has performed as many evaluations as it is allowed to.
      
      This method can be changed to true simulated annealing by varying the degree of
      modification made to the current best schedule with the number of evaluations
      remaining. Not being the focus of our investigation, little effort was made to 
      optimize the parameters. 1\% does seem to give better performance than other nearby 
      values, though we did not collect data to substantiate this. Finding ideal parameters 
      for simulated annealing on this problem would make for a project-sized inquiry in itself.

  \subsection{Options for the Genetic Algorithm}
    \subsubsection{Scheduling Problem Instances}
      We generated 2 sets of 3 scheduling problem instances to test the algorithms above with. 
      The package \texttt{ScheduleInstanceGenerator} in our code repository manages the generation of 
      random problem instances with many free parameters to specify characteristics of the 
      output instance.
      
      The first set of instances were generated with the parameters in the table below. 
      The rooms and courses and students were given arbitrary names and the courses that each 
      student attended were selected with uniform probability from the complete set of courses.
      
      The second set was generated with a sort of correlation between the courses that each student 
      took, rendering the dataset more true to the sort of data that would be found in real-world 
      instances of the exam scheduling problem. The set of courses that each student takes was 
      selected in the following way. Each student was assigned, with uniform probability, a \emph{major} 
      and a random number of courses between 2 and 6. Each course was then selected, with probability 
      0.7, from among the $|S|/\#Majors$ courses in their major, or, with probability 0.3, uniformly from 
      the set of all courses. We can expect that each student takes 70\% of their courses from a small 
      subset of the courses, representing the field that they specialize in, and 30\% from other fields. 
      This simulates the way in which we expect student bodies to choose their courses, though the true 
      data is likely to exhibit far more structure, due to program requirements, notorious professors, etc.
      
      The 3 sizes vary from manageable to large. The smallest could potentially be solved exactly, 
      with methods not under consideration here, while the others have search spaces much too large 
      for any hope of an exact solution. For the largest, which simulates a UofT exam schedule, a 
      single evaluation of the fitness function took 0.2 seconds on the fastest machine owned by any 
      of the authors.
      
      The following table characterized the problem instances that we used:
      
      \definecolor{light-gray}{gray}{0.87}
      \rowcolors{1}{light-gray}{white}
      \begin{center}
      \begin{small}
      \begin{tabular}{*{7}{c}}
      \hline 
      	& S 	& M 	& L 	& $S_{Cor}$ & $M_{Cor}$ & $L_{Cor}$ \\ 
      \hline \hline
       d 		& 7 	& 10	& 7 	& 7 	& 10 	& 7 	\\ 
       t 		& 5		& 8 	& 8 	& 5 	& 8 	& 8 	\\ 
       $|C|$	& 20 	& 200 	& 603 	& 20 	& 200 	& 600 	\\ 
       $|R|$	& 4 	& 10 	& 43 	& 4 	& 10 	& 43	\\ 
       $|S|$	& 50 	& 300 	& 21945 & 50 	& 300 	& 21945	\\ 
       $|L_s|$	& 1-5 	& 1-6 	& 1-5 	& 2-6 	& 2-6 	& 2-6 	\\ 
       \begin{tiny}$\#$Maj\end{tiny}	& N/A 	& N/A 	& N/A 	& 4 	& 10 	& 20 	\\ 
       \begin{tiny}\textsc{sss}\end{tiny}	& $10^{182}$ & $10^{1840}$ & $10^{6695}$ & $10^{182}$ & $10^{1840}$ & $10^{6689}$ \\
      \hline 
	  \end{tabular} 
	  \end{small}
	  \end{center}
	  
	  The large data set was created with parameters inspired from the University of 
	  Toronto's Faculty of Arts and Science.
	  The full input files for these instances are available in our repository in the 
	  directory \emph{Instance Input Files}.
	  
    \subsubsection{Selecting Parameters for the GA}
      [\ldots]
      
  \subsection{Comparison Between the Different Searches}
    [\ldots]
    \begin{figure}[h!]
  	  \centering
  	  \setlength\fboxsep{0pt}
	  \setlength\fboxrule{0.5pt}
	  \fbox{\includegraphics[width=8.35cm]{SMALL_FullAverageStats}}
      \caption{Search statistics on Small data set}
	\end{figure}
    
  
\section{Evaluation}
  \subsection{Success}
    [\ldots]
    Seeing as our fitness function gives a very high penalty for schedules that have direct
    conflicts (either a student with two exams at the same moment, or two different exams
    in a same room at the same moment), if such schedules exists given the input data, 
    our algorithms will tend to very quickly reject them. 
  
\section{Conclusion}
  [\ldots]
  We come up with a conclusion that in this particular problem, genetic algorithm
  is better than random search, and it is better than hill-climbing if the input 
  is a large data set. But it may not be the best solution for this problem, since
  the Mutate Search performs way better.
  
  \nocite{*}
\bibliographystyle{aaai}
\bibliography{paperBib}
\end{document}